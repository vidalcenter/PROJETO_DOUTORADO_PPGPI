
O desenvolvimento rural também pode ser compreendido por um conceito mais regional definido como: “Desenvolvimento Local”. Tal expressão é recente e deriva de iniciativas de mobilização organização social no sentido de promover uma maior representação dos diferentes atores sociais no processo de desenvolvimento. O Estado assume papel de agente facilitador desse processo de descentralização das políticas públicas para ser democrático, buscar transparência de suas instituições, o equilíbrio das forças exercidas pelas diferentes correntes de interesse e o compromisso com a qualidade de vida na população afetada \cite{castro_agricultura_2017}. O mercado econômico emergente, as necessidades de entregas urgentes e a redução cada vez maior das ofertas de emprego levou os centros de pesquisa e ensino a iniciarem o desenvolvimento de tecnologias voltadas a este setor, para que tal mercado se mantivesse ativo mesmo com o crescimento aumento da competitividade. 



Levando em consideração tais fatores, este trabalho relaciona-se estritamente com o Desenvolvimento Rural de forma sustentável e Circular. Em tempos passados, o conceito de Desenvolvimento Rural Sustentável era denominado por “Progresso Rural”, pois, havia um entendido genérico como sentido parcial e prático de “Melhoramento do ambiente” \cite{almeida_da_1995}. Entretanto, torna-se imprescindível destacar que, o desenvolvimento sustentável no meio rural não pode ter suas bases de compreensão apenas no progresso econômico, local ou regional.

As ações do homem agridem a natureza em quase todos os processos naturais, no entanto existem muitas atividades que usufruem das paisagens naturais sem agressão ao meio ambiente, como o turismo sustentável que explora, ou tem o compromisso de explorar a natureza selvagem, e exige respeito ao meio ambiente. Em outras situações como a atividade industrial, percebe-se que se retira da natureza sua matéria-prima e recursos energéticos, os quais são escassos e limitados, afetando sobremaneira o clima. Muitas vezes o processo tecnológico é imprevisível e os riscos e danos são invisíveis, porém refletem e agem silenciosamente no sistema ambiental \cite{marques_natureza_2017}.
