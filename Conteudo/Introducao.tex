\chapter{INTRODUÇÃO}



\section{JUSTIFICATIVA}

Esta pesquisa está focada na dissonância entre a teoria e prática do desenvolvimento tecnológico e as grandes e contínuas mudanças do meio rural. Este setor foi destacado por contribuir significativamente para a balança comercial do país, apresentando saldos positivos frequentes, setor que igualmente contribui para a segurança alimentar do País e produção de produtos limpos e renováveis \cite{Souza2014AgriculturaSaudaveis.}

O mercado emergente apresenta significativa contribuição para a empregabilidade da população no campo, invertendo cada vez mais o êxodo rural, porém, este mercado que absorve novos profissionais, apresente uma elevada demanda de novas tecnologias, desde equipamentos a novos tipos de cultivares tecnologias essas sejam capazes de lidar com o desenvolvimento tecnológico e a produção em larga escala \cite{deSouza2019ModernizacaoMaranhenses}. 

Atualmente a agricultura no Brasil principalmente a conduzida pelos pequenos agricultores apresenta-se como pilar da economia brasileira desde o período colonial, sendo o principal produtor dos alimentos mais consumidos, a exemplo: 70\% do feijão, 34\% do arroz, 87\% da mandioca, 46\% do milho, 38\% do café e 21\% do trigo. O setor de produção doméstica também é responsável por 60\% da produção de leite e por 59\% da carne suína, 50\% de aves e 30\% da bovina \cite{IBGE2017IBGERegionais}. 

A pandemia da COVID-19 agravou a insegurança alimentar nos centros urbanos devido à interrupção da cadeia de fornecimento de alimentos, por conta do agravamento das barreiras físicas e econômicas que restringem o acesso aos alimentos e ao aumento catastrófico do desperdício de alimentos devido à escassez de mão-de-obra \cite{Zurayk2020PandemicSecurity}. Desta forma, é necessário adotar sistemas de apoio aos produtores de alimentos locais que realizem uma distribuição mais eficiente e de menor custo, assim como reduzir o desperdício de alimentos durante a produção \cite{Brondeau2018TheMali}. 

Quanto à capacidade competitiva das pequenas propriedades que não possuem capacidade de investimento observa-se uma redução da competitividade ao longo dos anos, muito por conta do crescimento tecnológico do agronegócio, já que as grandes e médias empresas são capazes de investem várias áreas de inovação e marketing, assim como depositam valores significativos para pesquisa e desenvolvimento (P\&D) \cite{Kitchenko2019DecisionEquipment}. Se faz necessário, conhecer o cenário atual do desenvolvimento de Tecnologias  mais precisamente as de natureza industrial, direcionadas à agricultura familiar, de maneira que, decisões mais estratégicas sejam tomadas, buscando o engajamento aos avanços científicos aos produtores do setor agrícola, assim como a segurança alimentar, e promoção da agricultura sustentável.

O processo de transferência para as tecnologias agrícolas é extremamente importante e  estratégico para os pequenos produtores rurais que dependem de aportes governamentais, e as IES são potencialmente  capazes de dar um suporte a este fim. Os avançados processos de industrialização da agricultura exigem que os produtores menores não só conheçam suas potencialidades, mas que busquem parcerias de cooperação em universidades e centros de pesquisa, visando desenvolver suas capacidades produtoras e entrar no mercado competitivo atual \cite{Silva2016ModeloBrasileiros}.

A parceria Universidade/Produtor agrícola pode pode aumentar, de forma significativa, a capacidade de desenvolvimento das pequenas propriedades e diminuir o deficit de pequenos empreendimentos no campo, já que este setor vem apresentando expressiva queda se comparado ao crescimento de grandes propriedades rurais e latifúndios \cite{Culhane2016LearningEducation,Jabbour2019InstructorEducation,Gunasekera2018ExperiencesEducation}.

Desta forma, uma pesquisa científica que analise e desenvolva um modelo efetivo de transferência das tecnologias agrícolas produzidas nos Centros de ensino poderá apontar as principais barreiras impeditivas desse processo, além de fomentar desenvolvimentos científicos e tecnológicos futuros capazes de intervir positivamente no campo, uma vez que anteciparia futuros problemas relativos à aceitação da inovação produzida e possibilitando desta forma o desenvolvimento do homem rural em seu local de produção, o campo.

\section{OBJETIVO E METAS}

\subsection{OBJETIVO}
Desenvolver um modelo de transferência de Tecnologia e Inovação com enfoque nas demandas da agricultura familiar.

\subsection{METAS}

\begin{itemize}
\item{Caracterizar os processos de transferência de tecnologia agrícola realizada pelos órgãos competentes;}
\item {Levantar todas as variáveis envolvidas nos processos de transferência de tecnologia desde a  origem (IES e Centros de Pesquisa relacionados) até os ógãos responsáveis pela sua difusão;}
\item {Desenvolver o modelo para potencializar de transferência de tecnologia agrícola no âmbito IES-Centros de Pequisa-campo;}
\item {Validar o modelo de transferência de tecnologia junto aos especialistas;}
\end{itemize}


%\section{HIPÓTESE}




%%%%%%%%%%%%%%%%%%%%%%%%%%%%%%%%%%%%%%%%%%%%%%%%%%%%%%%%%%%%%%%%%%%%%%%%%%%%%%%%%%%%%%%%%%%%%%%%%%%%%%%%%%%%%%%%%%%%%%%%%%%%%%%%%%%%%%%%%%%%%%%%%%%%%%
                                                                 %REFERENCIAL TEÓRICO%                                                                             
%%%%%%%%%%%%%%%%%%%%%%%%%%%%%%%%%%%%%%%%%%%%%%%%%%%%%%%%%%%%%%%%%%%%%%%%%%%%%%%%%%%%%%%%%%%%%%%%%%%%%%%%%%%%%%%%%%%%%%%%%%%%%%%%%%%%%%%%%%%%%%%%%%%%%%