\chapter{INTRODUÇÃO}



\section{JUSTIFICATIVA}

Esta pesquisa está focada na dissonância entre a teoria e prática do desenvolvimento tecnológico e as grandes e contínuas mudanças do meio rural. Este setor foi escolhido por contribuir significativamente para a balança comercial do país, apresentando saldos positivos frequentes. Igualmente contribui para a segurança alimentar do País e produção de produtos limpos e renováveis. O mercado emergente apresenta significativa contribuição para a empregabilidade da população no campo, invertendo cada vez mais o êxodo rural, porém, este mercado que absorve novos profissionais, exige que tais profissionais sejam capazes de lidar com o desenvolvimento tecnológico e a produção em larga escala. 

Atualmente a agricultura no Brasil principalmente a conduzida pelos pequenos agricultores apresenta-se como pilar da economia brasileira desde o período colonial, sendo o principal produtor dos alimentos mais consumidos, a exemplo: 70\% do feijão, 34\% do arroz, 87\% da mandioca, 46\% do milho, 38\% do café e 21\% do trigo. O setor de produção doméstica também é responsável por 60\% da produção de leite e por 59\% da carne suína, 50\% de aves e 30\% da bovina. 

A pandemia COVID-19 agravou a insegurança alimentar nos centros urbanos devido à interrupção da cadeia de fornecimento de alimentos, ao agravamento das barreiras físicas e econômicas que restringem o acesso aos alimentos e ao aumento catastrófico do desperdício de alimentos devido à escassez de mão-de-obra. Assim, é necessário adotar sistemas de apoio aos produtores de alimentos locais e que realizem uma distribuição mais eficiente e de menor custo, assim como reduzir o desperdício de alimentos durante a produção. 

Quanto à capacidade competitiva das pequenas propriedades, observa-se uma redução ao longo dos anos, pelo crescimento tecnológico do agronegócio, em comparação às empresas que investem muitos em recursos, pesquisa e desenvolvimento (P\&D), com destaque para agricultura de precisão. Se faz necessário, conhecer o cenário atual do desenvolvimento de Tecnologias Intelectuais mais precisamente as de natureza industrial, direcionadas à agricultura familiar, de maneira que, decisões mais estratégicas sejam tomadas, buscando o engajamento aos avanços científicos aos produtores do setor agrícola, com maior consonância com o cumprimento de acabar com a fome, alcançar a segurança alimentar, melhorar a nutrição e promover a agricultura sustentável.

O processo de transferência para as tecnologias agrícolas pode ser extremamente importante e de modo estratégico para os pequenos produtores rurais que dependem de aportes governamentais, e as universidades se mostram capazes de dar um suporte a este fim. Os avançados processos de industrialização da agricultura exigem que os produtores menores não só conheçam suas potencialidades, mas que busquem parcerias de cooperação em universidades e centros de pesquisa, visando desenvolver suas capacidades produtoras e entrar no mercado competitivo atual \cite{silva_modelo_2016}.

A parceria Universidade/Produtor agrícola pode pode aumentar, de forma
significativa, a capacidade de desenvolvimento das pequenas propriedades e diminuir o déficit de pequenos empreendimentos no campo, já que este setor vem apresentando expressiva queda se comparado ao crescimento de grandes propriedades rurais e latifúndios. 



Desta forma, uma pesquisa científica que analise e desenvolva um modelo efetivo de transferência das tecnologias agrícolas produzidas nos Centros de ensino poderá apontar, antecipadamente, as principais barreiras impeditivas desse processo, além de fomentar desenvolvimentos científicos e tecnológicos futuros capazes de intervir positivamente no campo, uma vez que anteciparia futuros problemas relativos à aceitação da inovação produzida e possibilitando desta forma o desenvolvimento do homem rural em seu local de origem, o campo.

\section{OBJETIVO E METAS}

\subsection{OBJETIVO}
desenvolver um modelo conceitual de transferência de tecnologia agrícola que vise a efetiva condução de transferência de Tecnologia ao Produtor agrícola, principalmente o pequeno produtor.
\subsection{METAS}

\begin{itemize}
\item{Agregar os pontos subjetivos sobre a Transferência de Tecnologia agrícola a fim de promover um efetivo repasse para o usuário envolvido;}
\item {Averiguar o fluxo das transferências de tecnologia do Brasil realizadas por órgãos públicos por intermédio das publicações no Diário Oficial da União;}
\item {Compreender as estratégias governamentais, as iniciativas à promoção da transferência de tecnologia agrícola mediante a análise do atual momento;}
\item {Desenvolver uma ferramenta para avaliar a estrutura de transferência de
tecnologia agrícola no âmbito universidade-campo;}
\end{itemize}


%\section{HIPÓTESE}




%%%%%%%%%%%%%%%%%%%%%%%%%%%%%%%%%%%%%%%%%%%%%%%%%%%%%%%%%%%%%%%%%%%%%%%%%%%%%%%%%%%%%%%%%%%%%%%%%%%%%%%%%%%%%%%%%%%%%%%%%%%%%%%%%%%%%%%%%%%%%%%%%%%%%%
                                                                 %REFERENCIAL TEÓRICO%                                                                             
%%%%%%%%%%%%%%%%%%%%%%%%%%%%%%%%%%%%%%%%%%%%%%%%%%%%%%%%%%%%%%%%%%%%%%%%%%%%%%%%%%%%%%%%%%%%%%%%%%%%%%%%%%%%%%%%%%%%%%%%%%%%%%%%%%%%%%%%%%%%%%%%%%%%%%