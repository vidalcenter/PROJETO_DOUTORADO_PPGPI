\chapter{INTRODUÇÃO}



\section{JUSTIFICATIVA}

Esta pesquisa está focada na dissonância entre a teoria e prática do desenvolvimento tecnológico e as grandes e contínuas mudanças do meio rural. Este setor foi escolhido por estar contribuindo significativamente para a balança comercial do país, apresentando saldos positivos frequentes. Igualmente contribui para a segurança alimentar do País e produção de produtos limpos e renováveis. O mercado emergente apresenta significativa contribuição para a empregabilidade da população no campo, invertendo cada vez mais o êxodo rural, porém, este mercado que absorve novos profissionais, exige que tais profissionais sejam capazes de lidar com o desenvolvimento tecnológico e a produção em larga escala. 

Atualmente a agricultura no Brasil principalmente a conduzida pelos pequenos agricultores apresenta-se como pilar da economia brasileira desde o período colonial, sendo o principal produtor dos alimentos mais consumidos, a exemplo: 70\% do feijão, 34\% do arroz, 87\% da mandioca, 46\% do milho, 38\% do café e 21\% do trigo. O setor de produção doméstica também é responsável por 60\% da produção de leite e por 59\% da carne suína, 50\% de aves e 30\% da bovina. Quanto à capacidade competitiva, observa-se uma redução ao longo dos anos, pelo crescimento tecnológico do agronegócio, em comparação às empresas que investem muitos em recursos, pesquisa e desenvolvimento (P&D), com destaque para agricultura de precisão. Se faz necessário, conhecer o cenário atual do desenvolvimento de Tecnologias Intelectuais mais precisamente as de natureza industrial, direcionadas à agricultura familiar, de maneira que, decisões mais estratégicas sejam tomadas, buscando o engajamento aos avanços científicos aos produtores do setor agrícola, com maior consonância com o cumprimento de acabar com a fome, alcançar a segurança alimentar, melhorar a nutrição e promover a agricultura sustentável.

\section{OBJETIVOS}

\subsection{OBJETIVO GERAL}

Compreender as estratégias governamentais, as iniciativas à promoção da transferência de tecnologia agrícola mediante a análise do atual momento e posteriormente bem como desenvolver um modelo conceitual de transferência de tecnologia agrícola que vise a efetiva condução de transferências da Ciência Tecnologia e Inovação ao Produtor agrícola, principalmente o pequeno produtor
\subsection{OBJETIVOS ESPECÍFICOS}

\begin{itemize}
\item{;}
\item {;}
\item {;}
\item {}
\end{itemize}

\section{PROBLEMA}


\section{HIPÓTESE}




%%%%%%%%%%%%%%%%%%%%%%%%%%%%%%%%%%%%%%%%%%%%%%%%%%%%%%%%%%%%%%%%%%%%%%%%%%%%%%%%%%%%%%%%%%%%%%%%%%%%%%%%%%%%%%%%%%%%%%%%%%%%%%%%%%%%%%%%%%%%%%%%%%%%%%
                                                                 %REFERENCIAL TEÓRICO%                                                                             
%%%%%%%%%%%%%%%%%%%%%%%%%%%%%%%%%%%%%%%%%%%%%%%%%%%%%%%%%%%%%%%%%%%%%%%%%%%%%%%%%%%%%%%%%%%%%%%%%%%%%%%%%%%%%%%%%%%%%%%%%%%%%%%%%%%%%%%%%%%%%%%%%%%%%%