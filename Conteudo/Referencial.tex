\chapter{REFERENCIAL TEÓRICO}


\section{Inovação para o desenvolvimento rural}

O empreendedorismo quando relacionado à inventividade não consiste apenas na criação de uma empresa. Ser empreendedor vai além de ter seu próprio negócio e implica em uma visão de mundo diferente, uma mudança de paradigma, de pensamento. Garcia \cite{garcia_formacao_2000} define o empreendedorismo inovativo como sendo a habilidade de criar e de construir algo a partir do nada, sendo, desta forma, um ato humano nascido da criatividade. O empreendedor é um constante inovador, sendo aquele que está sempre em busca de soluções, que consegue enxergar nas oportunidades, tendo iniciativa e sendo proativo e visionário. Dessa forma, podendo contribuir para introduzir inovações na empresa na qual trabalha, ajudar a solucionar problemas da sua cidade, entre outras possibilidades que se fazem presentes a partir do momento que um comportamento empreendedor é desenvolvido, \cite{alencar_intencao_2019, loiola_cao_2016}.

A inovação, a propagação da inovação e o surgimento de novos empreendimentos, em muitos países, são importantes sinais para o crescimento e recuperação de crises econômicas \cite{silva_educacao_2017}. A inovação é orientada de acordo com várias racionalidades, podendo ser observada por diversas óticas e utilizando diversos instrumentos para aprendizado \cite{munoz_innovacion_2016}. Com efeito, o ambiente acadêmico se apresenta como uma unidade básica para o desenvolvimento de novos processos inovadores onde tais conteúdos devem ser amplamente explorados \cite{costa_inovacao_2017}. 

Desta forma, o desenvolvimento de uma visão empreendedora como ferramenta a inovação é essencial para formação de profissionais que tenham iniciativa, visão estratégica e capacidade de liderança, perfil este requerido pelas grandes empresas e startups. Além das características já citadas, é imprescindível para o empreendedor desenvolver sua criatividade, pois assim, torna-se mais fácil encontrar ideias originais e com valor \cite{macedo_capital_2019}. 

Para a inovação é necessário autoeficácia, pensar e agir diferente, encontrar soluções alternativas para os problemas e buscar ideias que tragam melhorias. Quando se fomenta a criatividade no ambiente educacional, a conquista da autonomia é consequência, assim como também a adoção de uma postura empreendedora \cite{gonzalez_predictors_2009}. Entretanto, na maioria das vezes, temos observado que essa importante característica não está sendo adequadamente desenvolvida no meio acadêmico. Ainda, as metodologias tecnicistas tradicionais de ensino exercem uma forte influência, as quais utilizam a transmissão de informações e concentram as atividades no docente. Tais posturas não vêm colaborando para o despertar da criatividade nos alunos, além de criar uma lacuna na formação profissional quanto ao desenvolvimento de competências essenciais. Porém este ambiente tem mudado, a partir do uso de metodologias ativas, as quais estimulam o reconhecimento dos problemas atuais, fortalecendo a criação de novos produtos, soluções e a dinamização de atividades diversas, se tornando uma oportunidade educacional de promoção ao empreendedorismo \cite{faria_promocao_2018}, potencializando a universidade para a criação de mecanismos que fomentem a manifestação dessas competências \cite{audy_innovation_2006}.

\section{Desenvolvimento do campo}

Definir o desenvolvimento rural sustentável requer um esforço observacional e prático, pois, este ambiente vem sofrendo profundas transformações em suas demandas e necessidades. O desenvolvimento que antes se apresentava majoritariamente como produção de subsistência, hoje dá lugar a um complexo sistema agroindustrial \cite{bastos_determinantes_2018} e social. É importante neste sentindo compreender que definir o desenvolvimento rural com apenas um conceito seria uma proposição simplista do contexto de crescimento no meio rural. Partindo da definição de consequência de ações governamentais definidas por \citeonline{navarro_desenvolvimento_2001} como “atitudes práticas”, este autor descreve que:

\begin{citacao}
[...] Desenvolvimento rural, portanto, pode ser analisado a “posteriori”, neste caso se referindo às análises sobre programas já realizados pelo Estado (em seus diferentes níveis) visando a alterar facetas do mundo rural a partir de objetivos previamente definidos. Porem pode se referir também à elaboração de uma ação prática.
\end{citacao}

O desenvolvimento rural também pode ser compreendido por um conceito mais regional definido como: “Desenvolvimento Local”. Tal expressão é recente e deriva de iniciativas de mobilização organização social no sentido de promover uma maior representação dos diferentes atores sociais no processo de desenvolvimento. O Estado assume papel de agente facilitador desse processo de descentralização das políticas públicas para ser democrático, buscar transparência de suas instituições, o equilíbrio das forças exercidas pelas diferentes correntes de interesse e o compromisso com a qualidade de vida na população afetada \cite{castro_agricultura_2017}.
 Neste contexto, surge as Organizações Não Governamentais (ONGs) que buscam garantir a participação da população local, e fazer valer tais mudanças atuando normalmente em ambientes geograficamente mais restritos (região rural, povoados ou municípios), \cite{assis_agricultura_2005, teixeira_o_2016}.


O mercado econômico emergente, as necessidades de entregas urgentes e a redução cada vez maior das ofertas de emprego levou os centros de ensino a iniciarem o desenvolvimento deste conteúdo disciplinar e os demais conteúdos relacionados. No início dos anos 80, o empreendedorismo estava diretamente ligado ao desenvolvimento econômico e à criação de postos de trabalho em um país \cite{rodrigues_intencao_2019}, passando a ser visto como importante fator a ser explorado nas comunidades acadêmicas. 



\section{Políticas públicas de apoio a Desenvolvimento tecnológico sustentável}

Este trabalho está relacionado com estudo do Desenvolvimento Rural de forma sustentável e circular. Anteriormente, o conceito de Desenvolvimento Rural Sustentável era denominado por “Progresso Rural”, pois, havia um entendido genérico como sentido parcial e prático de “Melhoramento do ambiente” \cite{almeida_da_1995}. Entretanto, torna-se imprescindível destacar que, o desenvolvimento sustentável no meio rural não pode ter suas bases de compreensão apenas no progresso econômico, local ou regional.

A compreensão em torno da pesquisa Científica, Tecnologia e Inovação (CT\&I) direcionadas à sustentabilidade ambiental no cenário nacional não se mantém constante ao longo da história do país. Fatores culturais e sociais dificultaram a relação das CT\&I e outros setores importantes do ambiente social, econômico e político do país. Inicialmente, as ações ambientais eram destinadas à elite econômica, dessa forma, os setores responsáveis se afastavam das demandas que permeavam a estrutura social do país. Atualmente, as ações ambientais são vistas como importantes fatores de contribuição para o desenvolvimento sustentado e sustentável de um país. Se de um lado, a Revolução Industrial apresentou um novo cenário para as atividades econômicas e humanas, os novos processos, tecnologias e produtos transformaram as concepções sobre os recursos naturais, o ser humano e o Planeta.

A Ciência, Tecnologia \& Inovação devem apresentar soluções com o objetivo de controlar os problemas socioambientais provocados pelo desenvolvimento predatório. O desenvolvimento sustentável e a responsabilidade social assumem um papel cada vez mais importante nas estratégias das empresas. O agravamento dos problemas ambientais está relacionado a maneira como o conhecimento técnico-científico tem sido aplicado no processo produtivo. Os danos ao meio ambiente não são fatos inesperados e imprevisíveis, apenas demonstram a falta de capacidade do conhecimento de controlar os efeitos gerados pelo desenvolvimento industrial \cite{maranhao_dinamica_2016}.

As ações do homem agridem a natureza em quase todos os processos naturais, no entanto existem muitas atividades que usufruem das paisagens naturais sem agressão ao meio ambiente, como o turismo sustentável que explora, ou tem o compromisso de explorar a natureza selvagem, e exige respeito ao meio ambiente. Em outras situações como a atividade industrial, percebe-se que se retira da natureza sua matéria-prima e recursos energéticos, os quais são escassos e limitados, afetando sobremaneira o clima. Muitas vezes o processo tecnológico é imprevisível e os riscos e danos são invisíveis, porém refletem e agem silenciosamente no sistema ambiental \cite{marques_natureza_2017}.

Para o desenvolvimento sustentável, a ciência e a tecnologia correspondem a um sistema de articulação entre uma racionalidade ambiental do processo de desenvolvimento e os processos concretos que definem as possibilidades de estratégias de manejo integrado do meio ambiente. Essa interação requer que o sistema de ciência e tecnologia esteja sustentado por paradigmas que incorporem o potencial ecológico, as condições ambientais e os valores culturais na organização dos processos produtivos \cite{furlan_gestao_2018}.

A busca por inovação de processos e de produtos possibilita o alcance dos níveis mais elevados de competitividade. Neste sentido, a literatura especializada reporta 

 que o desenvolvimento econômico de um país depende de uma integração qualificada de empresas, sustentabilidade ambiental e da economia mundial \cite{swinburn_desenvolvimento_2006}
. Para as empresas serem bem-sucedidas em mercados internacionais são necessárias vantagens competitivas baseadas em inovação, complexidade e/ou sofisticação dos produtos e um complexo sistema de compensação dos impactos ambientais que causam, isto é, devem serem inovadoras ou intensivas em conhecimento da economia circular \cite{lucas_desenvolvimento_2019}.



Nesse sentido, a inovação e tecnologia direcionadas para a sustentabilidade podem ser alternativas para redução de problemas ambientais causados pelo desenvolvimento industrial, que resultou no avanço da produção de novos produtos e serviços. Considerando que essa relação inovação, tecnologia e sustentabilidade sejam abrangentes, obtendo vários conceitos e definições como inovação sustentável, verde, eco ou ambiental, e tem por objetivo principal a redução de riscos ambientais, poluição e outros impactos negativos ao meio ambiente
\cite{pinsky_inovacao_2017}.  

Nos últimos anos, no Brasil, vários trabalhos vêm sendo desenvolvidos no campo das tecnologias sustentáveis com foco na minimização dos impactos ambientais. Estratégias como gerenciamento adequado dos resíduos sólidos e líquidos, uso de fontes energéticas renováveis e produção local dos alimentos vêm sendo cada vez mais desenvolvidas. As técnicas supracitadas auxiliam em um melhor desempenho dos índices sociais, econômicos e ambientais das populações quando relacionadas à sustentabilidade.

Estratégias de inovação tecnológicas concebidas dentro dos princípios do Desenvolvimento Sustentável e de Tecnologias Apropriadas (TA) poderão ser importantes na definição de tecnologias-chaves em que o país deva investir, tanto para a resolução dos seus problemas ambientais básicos, como para uma política de exportação de tecnologias, principalmente em países em desenvolvimento que contam com pouco capital para importar tecnologias caras de países industrializados. Para Teixeira \citeonline{teixeira_o_2002} as políticas públicas – tidas como diretrizes e princípios norteadores de ação do poder público – tem caráter estratégico para a condução de uma nação. Quando associadas à CT\&I, essas políticas são fator determinante para o crescimento econômico de um país, especialmente no que diz respeito à capacidade de produção de bens e serviços e a influência no padrão de vida de sua população \citeonline{corazza_caminhos_2004}.
As políticas públicas de apoio à pesquisa e tecnológica vêm sendo fortalecidas no Brasil, tanto líderes acadêmicos e políticos, quanto gestores de instituições, organizações, ou grandes corporações.  Há um consenso de que a pesquisa é um investimento fundamental para o desenvolvimento sustentável e para a melhoria da qualidade de vida dos povos \cite{bufrem_politicas_2018}.

Neste contexto, surge o papel de protagonismo no país dos Sistemas Nacionais de Ciência, Tecnologia e Inovação (SNCTIs), que primam pela integração contínua das políticas governamentais com as estratégias empresariais e visam o alinhamento com as práticas mais avançadas no mundo. Destacam-se como principais atores desse sistema quatro grupos, a saber: Entes Políticos (com órgãos descentralizados do Poder Executivo, do Poder Legislativo e da Sociedade); as Agências de Fomento (CNPq, CAPES, FINEP, BNDES, EMBRAPII e FAP); as empresas e as Instituições de Ciência, Tecnologia e Inovação (ICTs) \cite{mcti_estrategia_2016}, e; indiretamente as universidades públicas.

De acordo com Lei n° 10.973/2004, conhecida como Lei de Inovação, as ICTs são “órgão ou entidade da administração pública que tenha por missão institucional, dentre outras, executar atividades de pesquisa básica ou aplicada de caráter científico ou tecnológico” \cite{brazil_l1097304_2004}. ICTs públicas, em geral, e suas estruturas laboratoriais, em particular, são responsáveis por atender às demandas empresariais por soluções técnicas aplicadas ao desenvolvimento de novas tecnologias \cite{turchi_politicas_2017}.

Constata-se que a atual estrutura de apoio à CT\\&I foi bastante beneficiada após a entrada em vigor da Lei n° 13.243, conhecida com Novo Marco Legal da Ciência, Tecnologia e Inovação em 2016. Tal medida teve como prioridades: 

\begin{enumerate}
   \item  Promover a integração e cooperação de empresas privadas com o setor público;
   \item facilitar os processos administrativos e de gestão;
   \item disponibilizar fomento desconcentrado para o desenvolvimento de setores de ciência, tecnologia e inovação; e,
   \item promover segurança jurídica para empresas e ICTs \section{Inovação e transferência de tecnologia}

 \end{enumerate}



Considerando a importância de estudos desta natureza, tanto para a definição de prioridades e fomento de investimento no campo das políticas governamentais relacionadas ao desenvolvimento científico e tecnológico do país, como para sua implementação e avaliação no nível local, esta pesquisa se desenvolve tendo como objetivo compreender as estratégias governamentais, as iniciativas à promoção da sustentabilidade ambiental mediante o cenário mundial e a posição do Brasil mediante os centros, bem como apontar os principais entraves encontrados para aplicação dessas ações. Para alcançar tal finalidade, abordagem metodológica deste trabalho é de caráter exploratório de pesquisa documental, análise de dados primários e secundários e revisão da literatura sobre as ações políticas no Brasil direcionadas aos CT\&I para a sustentabilidade ambiental.


