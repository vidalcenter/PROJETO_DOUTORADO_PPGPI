% resumo em português
\setlength{\absparsep}{18pt} % ajusta o espaçamento dos parágrafos do resumo
\begin{resumo}




A aquisição de novas tecnologias tem se tornado uma prática comum para as empresas privadas do setor agrícola que em parte justifica os rápidos avanços no agronegócio brasileiro, embora não se saiba ao certo o quanto é gerado e transferido de inovação pelas IES e Centros de Pesquisa, ainda mais para a pequena produção, representado pela agricultura familiar, que demanda apoio por meio de programas governamentais, para essa intermediação. Faz-se necessário, então, a construção de um modelo que venha viabilizar a transferência de tecnologias agrícolas produzidas nas IES e centros de pesquisas, diminuindo as diferenças de acesso a tecnologias sustentáveis que a cada dia aumento entre a produção em grande escala (Agronegócio) e a pequena produção (agricultura familiar). Neste sentido, este estudo tem como desenvolver um modelo de transferência de Tecnologia e Inovação com enfoque nas demandas da agricultura familiar. Este estudo se utilizará de uma metodologia de natureza mista, inicialmente com exploração quantitativa e posterior de validação de caso. Será utilizado para avaliar, identificar e caracterizar determinantes das transferências de tecnologias desenvolvidas no setor público do Brasil, uma abordagem mista de análise exploratória por meio do método de \textit{Eletronic Data Scraping} (EDS) e posterior investigação de conteúdo textual utilizando o recurso de análise lexical das publicações contidas no Diário Oficial da União sobre tecnologias transferidas efetivamente ao produtor rural. No segundo momento será organizado os indicadores mais significativos para o desenvolvimento de um modelo preliminar, a qual passará por um processo de validação com especialistas da área através do método Delphi. Para a realização deste método serão necessárias duas rodadas. O modelo final possuirá 6 dimensões de análise para viabilidade: política, tecnológica, econômica/financeira, ambiental/sustentável, conhecimento e inclusão social, contemplando assim os indicadores e descritores necessários para validar efetivamente uma transferência de tecnologia. Nesta fase buscar-se-á desenvolver um modelo de transferência que poderá ser utilizando como instrumento de avaliação e planejamento da gestão pública sobre a viabilidade do uso desta tecnologia no campo.

\textbf{Palavras-chave}: Comercialização; Negócios sustentáveis; Propriedade Intelectual; Desenvolvimento rural, agricultura familiar.
\end{resumo}