% resumo em português
\setlength{\absparsep}{18pt} % ajusta o espaçamento dos parágrafos do resumo
\begin{resumo}




A aquisição de conhecimentos tecnológicos tem se tornado uma prática comum para as empresas privadas do setor agrícola desenvolvendo cada dia mais rápido e em maior volume, porém pouco tem sido transferido para os pequenos produtores, carecendo estes de apoio governamental para tal adequação ao cenário atual de produção no campo. Faz-se necessário, então, a construção de um modelo favorável a ocorrência da transferência de tecnologias agrícolas produzidas nos centros de pesquisas e ensino já que as ideias mais adaptáveis a este público normalmente estão ligadas à estes centros. Neste sentido, este estudo tem como tem como objetivo compreender as estratégias governamentais, as iniciativas à promoção da transferência de tecnologia agrícola mediante a análise do atual momento e ao final desenvolver um modelo conceitual de transferência de tecnologia agrícola que vise a efetiva condução de transferências da Ciência Tecnologia e Inovação ao Produtor agrícola, principalmente o pequeno produtor. Este estudo terá uma metodologia de natureza mista, inicialmente com exploração quantitativa e posterior de validação de caso. Pra tanto será utilizado para avaliar, identificar e caracterizar determinantes das transferências de tecnologias desenvolvidas no setor público do Brasil, uma abordagem mista de análise exploratória  por meio do método de \textit{Eletronic Data Scraping} (EDS) e posterior  investigação de conteúdo textual utilizando o recurso de análise lexical das publicações contidas no Diário Oficial da União sobre tecnologias transferidas efetivamente ao produtor rural. No segundo momento será organizado os indicadores mais significativos para o desenvolvimento de um modelo preliminar, a qual passará por um processo de validação com especialistas da área através do método Delphi. Para a realização deste método serão necessárias duas rodadas. O modelo final possuirá 6 dimensões de análise para viabilidade: política, tecnológica, econômica/financeira, ambiental/sustentável, conhecimento e inclusão social, contemplando assim os indicadores e descritores necessários para validar efetivamente uma transferência de tecnologia. Busca com esta fase desenvolver um modelo de transferência que poderá ser utilizando como instrumento de avaliação e planejamento da gestão pública sobre a viabilidade do uso desta tecnologia no campo.

\textbf{Palavras-chave}: Comercialização; Negócios sustentáveis; Propriedade Intelectual; Desenvolvimento rural.
\end{resumo}